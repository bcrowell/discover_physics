\mychapterwithopener{breakfast}{Breakfast table, by Willem Clasz. de Heda, 1631. A
variety of images occur in the painting, some distorted, as a result of both reflection and
refraction.}{Images}\label{ch:images}\index{images}\index{light!images}

Images are the main reason you care about light rays:
you want rays to paint images on your retina. It might seem as though
this business of images was very complicated, and to understand them
you'd have to memorize lots of facts about different image-producing
devices --- cameras, telescopes, microscopes, fun-house mirrors ---
but all these devices work according to a few simple principles, which
you already know.
You just need the rules from chapter
\ref{ch:raymodel} governing specular
reflection, and the rules of refraction which you discovered in lab.

The eye of the octopus is a striking example of the subject's underlying simplicity.
The last ancestor you had in common with an octopus was
an animal having only primitive vision, so your eye and the octopus's developed by parallel
evolution. Even though they evolved independently, they are
remarkably similar, because the structure of an eye is
dictated by the laws of physics.%
\footnote{Fundamentalists who perceive a conflict between evolution\index{evolution}\index{creationism}\index{evolution!creationism}
and their religion have claimed that the eye is such a
perfect device that it could never have arisen through a
process as helter-skelter as evolution, or that it could not
have evolved because half of an \index{eye!evolution of}eye
would be useless. Actually the evolution of the eye is
well understood. We humans have
a version of the eye that can be traced back to the
evolution of a light-sensitive ``eye spot'' on the head of
an ancient invertebrate. A sunken pit then developed so that
the eye would only receive light from one direction,
allowing the organism to tell where the light was coming
from. (Modern \index{flatworm}flatworms have this type of
eye.) The top of the pit then became partially covered,
leaving a hole, for even greater directionality (as in the
\index{nautilus}nautilus). At some point the cavity became
filled with jelly, and this jelly finally became a lens,
resulting in the general type of eye that we share with the
bony fishes and other vertebrates. Far from being a perfect
device, the vertebrate eye is marred by a serious design
flaw due to the lack of planning or intelligent design in
evolution: the nerve cells of the retina and the blood
vessels that serve them are all in front of the light-sensitive
cells, blocking part of the light. \index{octopus}Octopi and
other \index{mollusk}mollusks have a more sensible
arrangement, with the light-sensitive cells out in front.}


\mysection[0]{Location and Magnification}\label{sec:imagelocation}
\begin{envsubsection}{A flat mirror}\index{images!flat mirror}\index{images!location}
The ray diagram in figure \figref{flatmirrorimageb} implies something
strange and subtle, which you probably didn't fully absorb when you first
saw it on page \pageref{fig:flatmirrorimage}. To appreciate it, try this
experiment. First, bring this page so close to your face that it touches
your nose. You can't focus your eyes on it, because it's
too close. Now go in the bathroom and touch your nose against the mirror.
Surprisingly, you can easily see your own eyes in focus.
The experiment demonstrates that the image was not on the surface
of the mirror. It was behind the mirror, as implied by the ray diagram.
The reason you were able to focus on it was that it was twice as far away
as the mirror's surface.

\margup{-59mm}{\fig{flatmirrorimageb}{The flame's image is behind the mirror.}
\spacebetweenfigs
\fig{puddle}{The image is underground.}
\spacebetweenfigs
\fig{faceinmirror}{My nose reflects light rather than emitting it, but
the ray diagram is just like figure \figref{flatmirrorimageb}.}}
Of course there wasn't really any stuff behind the mirror: no light, and
certainly no face. Nevertheless, we say that the image is at a definite
point in space, behind the mirror. It's useful to say this, because the only
way you can see things is if light rays go into your eyes, and when you catch
those rays, they don't carry any history. We know the rays in 
figure \figref{flatmirrorimageb} came from a point on the
actual flame on the left, but they form exactly the same spreading pattern
that would have been produced by a flame behind the mirror. It doesn't matter
whether the object emits light or merely reflects it, as in figure \figref{faceinmirror}.
The tip of my nose reflects light diffusely, so light rays move away from it in every direction,
just as they do from a point on the flame.

Summarizing, we can define an image like this:
\begin{important}
An image is where rays originating from the same point on the object either
cross again or appear to have crossed.\index{image!defined}
\end{important}
\noindent Figure \figref{faceinmirror} shows the case where the rays only appear
to have crossed at the image's location. Burning ants with a magnifying glass
is an example of the other case: rays that originated from the same point
on the sun are actually reunited on the ground.

An image's location is important, as exemplified by the famous warning message
on cars' rear-view mirrors, ``objects are closer than they appear.'' From the
ray diagrams in figures \figref{flatmirrorimageb} and \figref{faceinmirror}, it's
clear that a flat mirror produces an image whose distance behind the mirror is
equal to the object's distance in front of it. Therefore, when we increase an
object's distance from a flat mirror, the image's distance from the mirror increases
as well. We represent this with the shorthand symbol $\upup$.
(This is what high school
geometry teachers refer to disparagingly as proof by drawing, but proof by drawing
works, and we're going to do a lot of it.) 
Although it was easy to see that a
flat mirror would be $\upup$, the images made by lenses and curved mirrors are
sometimes $\upup$ and sometimes $\updown$, and we need to discuss a couple
of techniques that are more generally useful for determining this.

\widefig{frontbackmethod}{Another way of determining that the image is $\upup$. 1. Find a ray
that can get to the observer's eye without being blocked. 2. Rays from the paper airplane's
tail are not blocked,
so the tail is visible, not the nose. Therefore the image must be pointed away from the observer.
3. If the plane got farther from the mirror, the image would too, so the image is $\upup$.}
%
One method is simply to draw ray diagrams for two different object distances.
In the ray diagrams up to now, I've been drawing several rays, all of which
appeared to come from the image point. There is really no need to do more than
two such rays. We can also make life easy by choosing one ray to be the one
that happens to head straight towards the mirror; this ray is reflected right
back on itself. The full technique is demonstrated in figure
\figref{flatmirrorchangedistance}. This method is straightforward for images made
by flat and curved mirrors, but somewhat more cumbersome with lenses, both because
refraction is more complicated than reflection and because a ray passing through a
lens undergoes two refractions, one at each surface. Figure \figref{frontbackmethod}
is an alternative.
\end{envsubsection}
%
\begin{envsubsection}{A curved mirror}\index{images!curved mirror}\index{images!magnification}
There are only a few uses for a flat mirror. Figure \figref{inbendingmirrorvirtual} shows
the more interesting case: an image made by a curved mirror. (The figure only shows a section through
one plane, not all three dimensions; we'll assume throughout this book that our mirrors are
symmetric with respect to rotation about a central axis, like a dish, not a saddle or a potato chip.)
Because the mirror is curved, the reflected rays are bent back inward a little, and are not
diverging as strongly as the incident ones were.

\margup{-107mm}{\fig{flatmirrorchangedistance}{The object is moved from its original position
(heavy lines) to a new position farther from the mirror (light lines). For the ray
that strikes the mirror at an angle, it's helpful to draw in the normal (perpendicular)
to the mirror's surface. Since $r=i$, the new rays must fan in closer to the normal.
As the object's distance increases, so does the image's, $\upup$.}
\spacebetweenfigs
\fig{inbendingmirrorvirtual}{A curved mirror.}
}
%
The ray diagram in figure \figref{inbendingmirrorvirtual} shows that the image is farther from
the mirror than the object. The techniques described above show that the image is
still $\upup$, as with a flat mirror, but if we increase the object's distance from the
mirror by, say, 1 cm, then the image's distance will increase not just by 1 cm but by
some greater amount. All the front-back distances in Image-Land behind the mirror
have been magnified. What about up-down distances, and distances into and out of the page?
If you try drawing rays and locating the images of other points on the face, you'll find that
all the distances are enlarged consistently, which means that the image-face has all the same
proportions. In this particular example, the magnification is about two: all distances are
doubled.
\end{envsubsection}
%
\vfill
%
\mysection{Real and Virtual Images}\index{images!real and virtual}\index{real image}\index{virtual image}
The image in figure \figref{inbendingmirrorvirtual} is $\upup$, but if we keep increasing
the object distance, an increase in the image distance isn't the only effect we'll ever see.
Eventually we'll get the completely different situation shown in figure \figref{inbendingmirrorreal}. 
In figure \figref{inbendingmirrorvirtual}, the cone of rays intercepted by the mirror was spreading
out strongly, and although the mirror bent them back in somewhat, the reflected rays were still spreading.
In figure \figref{inbendingmirrorreal}, the rays coming out at the biggest angles miss the mirror
entirely, and those that do reach the mirror form a cone that isn't diverging so
strongly. The mirror is able to bend them enough so that they reconverge.
This switch in behavior occurs when the object is at a certain distance from the mirror,
called the mirror's focal\index{focal length}\label{focal-length-definition}
length.\footnote{The interpretation of the focal length is elaborated on in
homework problem \ref{hw:interpretfocallength} and lab \ref{ch:images}\ref{lab:telescope}.}

\margup{-50mm}{\spacebetweenfigs
\fig{inbendingmirrorreal}{A new kind of image.}}
The image point in figure
\figref{inbendingmirrorvirtual} is one where the rays only appear to have crossed; this is called
a virtual image. In figure \figref{inbendingmirrorreal} the rays really do cross at the image point,
and such an image is referred to as a real image. Only a real image can be projected onto a screen.
A movie projector, an overhead projector, and the human eye all form real images. (The eye's
``screen'' is the retina, a layer of light-sensitive cells connected to the brain by nerves.)


\selfcheck{virtualtoreal}{%
Starting with the object distance shown in figure 
\figref{inbendingmirrorvirtual}, suppose we gradually move the object farther and farther away from the mirror.
At some special object distance, the image changes from virtual to real. What do the reflected rays look like
in this special case?%
}

\selfcheck{realisupdown}{%
Use the methods shown in figures \figref{frontbackmethod} and
\figref{flatmirrorchangedistance} to determine whether the image in figure
\figref{inbendingmirrorreal} is $\upup$ or $\updown$.
}

\selfcheck{realmagnification}{%
The image in figure \figref{inbendingmirrorreal} is smaller than the
original object. Why smaller, and not bigger as in figure \figref{inbendingmirrorvirtual}
on page \pageref{fig:inbendingmirrorvirtual}?
}

\vfill
\mysection{Angular Magnification}\index{images!magnification!angular}\index{magnification!angular}
Figure 
\figref{telescope} 
shows a real-life application of these concepts: a
telescope. The curved mirror at the bottom of the tube forms a real image.
Some professional telescopes are so large that a camera or even an astronomer's
body can fit inside the tube. That's not possible with a smaller telescope like this,
so a small second mirror is used. This flat diagonal mirror makes an image of the
image, outside the tube where we can see it. (The setup shown in the figure
will work, and is what amateur astronomers use for astronomical photography.
When observing by eye, however, one usually includes a third optical element, an eyepiece lens, for
greater magnification.)

\marg{\fig{telescope}{The images formed by the telescope are smaller than the
moon itself.}\label{telescope-figure}}\index{telescope}
What's interesting about this example is that although we think of a telescope
as a device for magnifying astronomical objects, the images are \emph{smaller} than the
actual moon: a few centimeters across, compared to thousands of kilometers for the
original object. The magnification is therefore a number much smaller than one,
perhaps on the order of 0.000001. Rather than magnifying the moon, the telescope
shrinks or ``minifies'' it. Why, then, do our eyes tell us that the image is
\emph{bigger} than the moon? It's because closer objects appear larger. The actual
moon is much larger than the image, but it's also millions of times farther away.
Because the image is outside the telescope, you can move your head as close to the
image as you like. The only limitation is that your eye can't focus on objects that
are less than a few centimeters away.

This shows that in many situations, it isn't magnification that we care about but
angular magnification:
\begin{align*}
	\text{magnification}	&= \frac{\text{size of image}}{\text{size of object}}\\
	\text{angular magnification}	&= \frac{\text{angular size of image}}{\text{angular size of object}}
\end{align*}
The reason a distant mountain looks smaller than a nearby tree is that your eye can
only tell you the angular size of an object, not its actual size. The telescope gives
a magnification much less than one (extreme ``minification''), but an angular magnification
much greater than one (typically from 20 to 500).
\vfill
%===============================================================================
%===============================================================================
\begin{hwsection}

\begin{hw}{abnormals}
The figure shows the cross-section of a funhouse mirror. Some of the normals
are correct, and some are incorrect. Print out a copy of the figure, and mark
the correct and incorrect ones. Fix the ones that are incorrect.
\end{hw}

\margup{-15mm}{%
\fig{abnormals}{Problem \ref{hw:abnormals}.}
\spacebetweenfigs
\fig{idoesntequalr}{Problem \ref{hw:idoesntequalr}.}%
\spacebetweenfigs
\fig{black-box}{Problem \ref{hw:black-box}.}
}
\begin{hw}{idoesntequalr}
Which reflections are correct, and which are incorrect? Correct the ones that are wrong.
\end{hw}

\begin{hw}{invertedforehead}
In figure \figref{inbendingmirrorreal} on page \pageref{fig:inbendingmirrorreal},
only the image of my lips was located by drawing rays. Print out a copy
of the figure, and trace a new
set of rays coming from my forehead. By comparing the locations of the image of the
lips and the image of the forehead, demonstrate that the image is in fact upside-down
as suggested by the figure.
\end{hw}

\begin{hw}{walkingtowardmirror}
A woman is walking directly toward a flat mirror at 1.0 m/s. At what rate is
her distance from her image decreasing?
\end{hw}

\begin{hw}{reflectiononfloor}
Walking down a long corridor at 1.0 m/s, you notice that the shiny floor is forming
a reflection of a light fixture that is mounted on the ceiling ahead of you. \\
(a) Does the image move? If so, at what speed is it moving? \hwendpart
(b) What is the closest you ever get to this image? Draw a ray diagram to 
locate the image at the point where the rays cross or appear to have crossed.
Does this make you change your mind about your answer
to part a?
\end{hw}

\begin{hw}{mirrorleftright}
People say that mirrors switch left and right. Is this really true? The following are some
suggestions for definite, specific examples that you may find it helpful to think about.
If you face south into
a mirror, and point your finger to the east, consider whether your image points east or west,
and likewise think about the case where you're pointing in other directions, such as up, or
south. What would a mirror
on the ceiling over your head do? What about a mirror that was on your left, catching
your profile?  State a general rule.
\end{hw}

\begin{hw}{black-box}
The figure shows two mysterious devices that do something to light rays that pass through them.
You don't know what kinds of mirrors or lenses might be inside these two black boxes, but you
are able to observe what they do to the rays, as shown in the drawings. 
Copy the drawings onto your paper. Locate the images, and classify them as real or virtual. Which
device could be used in the same manner as the kind of overhead projector used in classrooms?
\end{hw}


\begin{hw}{interpretfocallength}\index{focal length!interpreted as strength}
Does a more strongly curved mirror have a shorter focal length, or a longer one? Explain
using a ray diagram,
making explicit use of either the definition on page
\pageref{focal-length-definition} or the one on page \pageref{alt-focal-length-definition}.
\end{hw}

\begin{hw}{possiblemagnifications}
Consider a converging mirror, i.e., one whose hollowed-out side is silvered.
(a) Can the magnification of a real image formed by such a mirror ever be greater than one?\hwendpart
(b) Can the magnification of a virtual image formed by such a mirror ever be a number greater than
zero but less than one?\hwendpart
If you answer yes to one of these questions, give an example with a ray diagram to prove
that you're right. If you answer no, explain why it's not possible.
\end{hw}

\pagebreak[4]

\begin{hw}{outbendingupdown}
All of the curved mirrors shown in this chapter were hollowed out, like a dish,
but it's also possible to have a mirror that bulges outward.\\
(a) Draw a ray diagram to show
how an image is formed by such a mirror. Your answers to the rest of the questions
should be explained by referring to this ray diagram. (You may want to add to it
or draw additional ones.)\hwendpart
(b) Is the image real, or virtual?\hwendpart
(c) Is the magnification less than one or greater than one? Explain how you can tell
from the ray diagram.\hwendpart
(d) If you increase the object distance, what happens to the image distance?
Make explicit use of one of the two methods discussed in section \ref{sec:imagelocation},
and show your work.
\end{hw}

\hwnote{%
Problems \ref{hw:lens-underwater}-\ref{hw:goggles}
are to be done after you've completed all the labs, and know about refraction and
lenses. A converging lens\index{lens!converging!defined} is one that tends to bend
light rays closer together. A typical converging lens is a piece of glass or plastic that's
thickest in the middle, like an M\&M. A diverging lens\index{lens!diverging!defined}
tends to spread rays apart, and is thinnest in the middle.
}

\begin{hw}{explain-lens}
Based on the rules you've learned for refraction, explain why light rays passing
through the edges of a converging lens are bent more than ones that pass through parts
closer to the center. It might seem like it should be the other way arond, since the rays
at the edge pass through less glass --- shouldn't they be affected less? As part of
your explanation, draw a big close-up ray diagram showing the cross-section of the lens.
\end{hw}

\begin{hw}{lens-underwater}
Suppose a converging lens is made out of a material
in which the speed of light is less than in air, but greater than in water. How will the
lens's behavior be different if it's placed underwater?
\end{hw}

\begin{hw}{lens-to-film}
When you focus your camera on something farther away, does the lens have to move farther from
the film, or closer to it? Explain.
\end{hw}

\begin{hw}{misinterpretfocallength}
In your answers to both part a and part b, give full explanations,
making explicit use of either the definition of focal length given on page
\pageref{focal-length-definition}, or the
one on page \pageref{alt-focal-length-definition}.\hwendpart
(a) Is the focal length of a lens a fixed property that could be permanently stamped on it,
or does it depend on how you use it?\hwendpart
(b) In a camera, does the distance from the lens to the film equal the lens's focal
length?
\end{hw}

\margup{-55mm}{\fig{eye-cross-sections}{Problem \ref{hw:goggles}.}}
\begin{hw}{goggles}
When you're swimming underwater, why is it that you can see much more clearly when you're
wearing goggles consisting of flat pieces of plastic
that trap air in front of your eyes? Give an explanation that includes a ray
diagram. For simplicity, consider the case where the object you're looking at is very far away,
and lies along the optical axis (i.e., the line perpendicular to the goggles).

The figure may help you to understand how the human eye works under normal conditions.
The first drawing is a realistic cross-section of the eye.
For our purposes, however, it will be sufficient to consider the simplified version shown
below: a ball of clear jelly with a bump on it. Light passes into the
eye through the bump, and almost all the refraction happens at that point. (The small
interior ``lens'' is really only a secondary fine-adjustment device --- it doesn't bend the rays
of light very much, because the speed of light in it is not very different from the speed of
light in the other jellylike substances that surround it.)
\end{hw}


\end{hwsection}

%========================================== labs ===============================================
%--------------------------------- images lab --------------------------------------

\begin{lab}[0]{Images}\label{lab:images}

\apparatus
\equip{plastic box with water in it}\\
\equip{laser}\\
\equip{ruler}\\
\equip{protractor}\\
\equip{paperclips}

\goal{Locate an image in a tub of water by ray tracing,
and compare with its location as measured by eye using parallax.}

\labpart{Parallax}

The figures show the basic idea of the lab. When you view the setup
from the side, you're seeing an image of the submerged pointer, not the
pointer itself. This is an example of an image formed by refraction rather than
reflection.
By closing one eye and then the other, you can see the
parallax of each pointer. By moving the top pointer, you can get it to have
the same parallax as the image of the submerged one, which means it's at the same
distance as the image of the submerged one. Parallax is strongest when you're
as close as possible to the object, so put the tub of water at the edge of the
desk, and crouch down with your face very close to it.

\labfigcaption{lab-image-top-view}{Make two pointers of different heights out
of paperclips. The taller pointer's tip is above the
water, while the shorter one's is submerged.}

\labfigcaption{lab-image-side-view}{The same setup viewed from the side.
One pointer is directly above the other.}

Measure the object and image distaces from the front surface of the tub.
When I did this, I was able to locate the image's position to within about
a millimeter, and I got good agreement between the parallax method, opening
one eye at a time, and depth perception, with both eyes open.

\labpart{Changing the location}

Make a ray diagram, showing how each ray moves through the water, is refracted, and
goes off through the air.
Use one of the two methods described in section \ref{sec:imagelocation} to predict
whether the image should be $\upup$ or $\updown$. 
If you have time, do the other method too, and verify that you get the same
answer.

\labfigcaption{lab-spreading}{Light rays spread out from the finger by
diffuse reflection. The emerging rays all appear to have come from a point
inside the box.}

Now check your prediction by taking data at a different object distance.
It will be convenient if you do this by
substituting your finger, pressed against the back of the tub, for the submerged
pointer. Your data should now consist of two object distances and two image
distances.

\labpart{Ray tracing}
Now you're going to see if you can reproduce the image location from
part B by ray tracing.
Trace the outline of the box on a piece of paper,
remove the box, mark the location of the image, and put
the box back on the paper. Shine the laser at the point
where your finger was originally touching the box, observe
the refracted beam, and draw it in. Repeat this whole
procedure several times, with the laser at a variety of
angles. Finally, extrapolate the rays leaving the box back
into the box. They should all appear to have come from the
same point, where you saw the image.


\labfigcaption{lab-ray-tracing}{Simulating one of the rays using the laser.}

\vfill

\end{lab}

%--------------------------------- real image lab --------------------------------------

\begin{lab}{A Real Image}\label{lab:real-image}

\apparatus
\equip{concave mirror and holder}\\
\equip{pointer}\\
\equip{illuminated object}\\
\equip{optical bench}

\begin{goals}

\item[] Observe a real image formed by a curved mirror.

\item[] Make qualitative observations of the image and explain them using ray diagrams.

\end{goals}

\labpart{Initial observations}

Put the mirror in the holder and put the holder in the clamp that holds it on the optical
bench. Right now you're just using the hardware as a hands-free way to hold the mirror in
position. Your mirror may be silvered on both sides; you're going to be using the hollowed-out
(concave) side.

Standing a couple of meters away, look at the reflection of your own face in the
mirror. Now move your face closer and closer to the mirror, and observe the changes that
occur. 

\labpart{Distant, fixed object}

Part A was a quick and dirty way to get acquainted with what's going on, but it's a little
complicated to understand, because as you move closer to the mirror, you're moving both
the object (your face) and your point of view. Let's now try some observations in which
you leave an object in one place, and observe it from different points of view.
Position yourself and the optical bench so that, from a distance of a couple of meters,
you can look at the mirror and see the reflection of something that's behind you, over
your shoulder and far behind you. Move closer and closer, while observing the image
of the stationary object behind you. 
You'll see various changes in the image, but let's concentrate on one thing: when the
image is clearly visible and when it's impossible to focus on it.

\pagebreak[4]

Draw a ray diagram to show how this image is being formed:

\newcommand{\labspaceforraydiagram}{\vspace{30mm}}
\labspaceforraydiagram

Use your ray diagram to explain your observations.
Don't go on to the next part until you understand the observations you've made. Ask your
instructor for help if necessary.

\labpart{Close, fixed object}

Now repeat part B, but with an object only 5 or 10 cm from the mirror. The most convenient
way to do this is to stick the upright pointer in the optical bench, so your hands are free.

\labspaceforraydiagram

Again make sure you can explain your observations in terms of your ray diagram.
You'll have found that there's a difference between parts B and C in terms of the eye positions
from which you can't see the image, or can't see it clearly; make sure you understand why this is.

\labpart{Moving object}
Replace the pointer with the illuminated object, and slide it all the way to the far end
of the optical bench, so it's as far from the mirror as it can be. By putting a small piece of paper at the right
point in space, you should be able to get the mirror to project an image of the object onto
the paper. Note that although you want everything located approximately along the line
of the optical bench, you don't want the paper to block all the light from getting from
the object to the mirror. To avoid this, you may want to angle the mirror a tiny bit, and
put the paper a tiny bit off to the side. Draw a ray diagram, and indicate on the diagram
the special point in space where you can put the paper in order to see a clear image.

\labspaceforraydiagram

Now move the object closer to the mirror. What do you have to do to get a clear image again?
Move the object closer and closer to the mirror, and keep going as far as you can with this
setup.

Is your image $\upup$ or $\updown$? Explain this observation with a ray diagram similar to
figure \figref{flatmirrorchangedistance} on
page \pageref{fig:flatmirrorchangedistance}.

\labspaceforraydiagram

\labpart{You're in my light!}
Imagine --- but don't do it yet! --- that with the setup from part D, you cover half of the
mirror with your hand. What effect do you think this would have on the image? To make
your prediction, use your ray diagram.

\newcommand{\labprediction}{prediction:\_\_\_\_\_\_\_\_\_\_\_\_\_\_\_\_\_\_\_\_\_\_\_\_\_\_\_\_\_}
\labprediction

Now try it. If your prediction was wrong, figure out what happened.

Now what do you think would happen if you covered half of the object?

\labprediction

Again, try it, and, if necessary, back up and think again.

\labpart{Magnification (optional)}
With a setup like the one in part D, measure the magnification of the image produced at three
different object distances: the two extreme ones plus one in the middle. What trend do you
observe, and why does it occur?

\end{lab}

%--------------------------------- lenses lab --------------------------------------

\begin{lab}[0]{Lenses}\index{lens!lab}

\apparatus
\equip{converging and diverging lenses}\\
\equip{illuminated object}\\
\equip{optical bench}

\begin{goals}

\item[] Find all the types of images that can be made with lenses, and explain each
type using a ray diagram.

\item[] Use ray diagrams to predict whether a real image made by a lens
 is $\upup$ or $\updown$, and
test your prediction.

\end{goals}

Like mirrors, lenses come in two types: a converging type that brings rays together
and a diverging type that spreads them apart. By convention, converging lenses are
described with positive focal lengths and diverging lenses with negative ones.
As with mirrors, it's also sometimes possible
for the same lens to make either a virtual or a real image, depending on the object distance.

The basic setup for this lab is like the one used in lab \thechapter\ref{lab:real-image}. However, some of
your images will be virtual, which means you can't project them onto a piece of paper.
When you get a real image, take numerical measurements to show how the changing the object distance
affects the image distance ($\upup$ or $\updown$), and check this against what you find by
drawing ray diagrams with different object distances.
When you get a virtual image, just draw a ray diagram showing what's going on.

You should get a total of three qualitatively different types of image formation. That is,
in principle you could use either a converging lens or a diverging one, and you could
use either one to make either a real or a virtual image, resulting in a total of four
possibilities. However, one of these turns out not to be possible, so you'll end up with
only three cases.



\end{lab}

%--------------------------------- telescope lab --------------------------------------


\begin{lab}{The Telescope}\index{telescope}\label{lab:telescope}

\apparatus
\equip{optical bench}\\
\equip{lens, longest available focal length}\\
\equip{lens, 50 mm focal length}\\

\begin{goals}

\item[] Construct a telescope.

\item[] Measure its angular magnification, and compare with theory.

\end{goals}

\labintroduction

The credit for invention of the telescope is disputed, but
Galileo was probably the first person to use one for
astronomy. He first heard of the new invention when a
foreigner visited the court of his royal patrons and
attempted to sell it for an exorbitant price. Hearing
through second-hand reports that it consisted of two lenses,
Galileo sent an urgent message to his benefactors not to buy
it, and proceeded to reproduce the device himself. An early
advocate of simple scientific terminology, he wanted the
instrument to be called the ``occhialini,'' Italian for
``eye-thing,'' rather than the Greek ``telescope.''

His astronomical observations soon poked some gaping holes
in the accepted Aristotelian view of the heavens. Contrary
to Aristotle's assertion that the heavenly bodies were
perfect and without blemishes, he found that the moon had
mountains and the sun had spots (the marks on the moon
visible to the naked eye had been explained as
atmospheric phenomena). This put the heavens on
an equal footing with earthly objects, paving the way for
physical theories that would apply to the whole universe,
and specifically for Newton's law of gravity. He also
discovered the four largest moons of Jupiter, and demonstrated
his political savvy by naming them the ``Medicean satellites''
after the powerful Medici family. That they
revolved around Jupiter rather than the earth helped make
more plausible Copernicus' theory that the planets did not
revolve around the earth but around the sun. Galileo's ideas
were considered subversive, and many people refused to look
through his telescope, either because they thought its images were
illusions or simply because it was supposed to show things
that were contrary to Aristotle.

\labfigcaption{lab-telescope}{A refracting telescope. The rays
coming from the object first encounter a relatively weak lens,
called the objective. An intermediate real image is formed, but
what your eye sees is an image of the image, created by the
eyepiece lens. The solid lines are two rays that are both
coming from the same point in the sky; because the celestial object
is so far away, they're essentially parallel. The dashed lines
are coming from some other point. The angles are exaggerated in order to
demonstrate the angular magnification, and because of this, the
solid-line rays aren't even going to get into the person's eye.}

\vfill\pagebreak[4]
\labsubsection{Why It Works}

The figure above
shows the simplest refracting telescope.
The point of the whole arrangement is angular magnification.
The small angles on the left are converted to large angles
on the right, because the eyepiece is more strongly curved,
and therefore bends the rays more. The strength of a lens
is measured by its focal length (homework problem
\ref{hw:interpretfocallength}).
For example, if the ratio of the two lenses' focal
lengths is eight, then the eyepiece
bend the rays eight times as much, and the angular magnification
will theoretically equal eight. To get the maximum angular magnification,
you want the eyepiece to be as strong as possible, and the objective
as \emph{weak} as possible! (Remember, dividing by a small number gives
a big result.)

Here's a second, alternative way of thinking about it. The objective
creates a real image. This image is located near you, where you can increase
its apparent size simply by getting close to it. In fact, it's possible to
use the telescope without an eyepiece at all! However, now that you've got
this nice convenient nearby image, you can also magnify it some more by
looking at it through a magnifying glass, just like any other small, nearby
thing. The eyepiece is the magnifying glass. This makes it clear why a strong
eyepiece lens gives a greater magnification, but why does a weaker objective
lens give a greater magnification as well? Well, the size of an image is always
proportional to its distance from the lens making it. The longer the focal length
of the objective, the greater the distance from it to the real image, so we conclude
that a longer focal length for the objective (a weaker lens) gives a bigger image,
which will appear even bigger when viewed through the eyepiece.

\labobservations

\labpart{The focal lengths}

To start with, let's try to get a feel for the physical meaning
of the focal length. Use one of the lenses to project an image
of the overhead lights onto the floor. If we make the approximation
that the overhead lights are infinitely far above, then the distance
from the lens to the image equals the lens's focal length.\index{focal length!alternative definition}\label{alt-focal-length-definition}
This is different from the definition of the focal length given on page
\pageref{focal-length-definition}. To see that they're equivalent definitions,
take a look at the figure below.

\labfig{lab-focal-length}

If we interpret the point on the right where the rays cross as the object, then the
image is infinitely far off to the left. This corresponds to the original definition
of the focal length: the cross-over point between real and virtual images.
If the object was a little closer, then the rays on the left would be diverging slightly,
and the image would be a virtual one far off to the right. If the object was a little farther,
then the rays on the left would be converging a little, and there would be a real image
very far to the left.

However, the laws of physics have time-reversal symmetry, so if the diagram is valid for
rays traveling from right to left, it's also valid for rays going left to right. In this
case, the object (think of the overhead lights) is infinitely far off to the left,
and the point on the right where the rays cross is the location of the real image
(projected on the floor in our case).

By projecting the image of the overhead lights onto the floor, verify the
focal lengths printed on their plastic housings.

\labpart{The telescope's magnification}

Use your optical bench and your two lenses to build a
telescope. Take the data you will need for a rough
determination of its angular magnification. One easy method
is to observe the same object with both eyes open, with one
eye looking through the telescope and one seeing the object
without the telescope.

If you find that you can't focus on both things at once,
try making small adjustments to the distance between the
lenses. The reason this problem can occur is that neither
the focal lengths printed on the lenses nor the focal
lengths you measured in part A are terribly accurate,
so the distance between the lenses isn't quite what it
should be. The rays coming to your eye are therefore not
quite parallel, which means that the image they form is
not at infinity. Your body is not capable of simultaneously 
focusing one eye at infinity and one at a short distance.
\end{lab}

%========================================== self-check answers ===============================================



\startscanswers{ch:images}
\scanshdr{virtualtoreal}{The reflected rays are parallel. This can be interpreted by saying that
the image is at infinity: as you make two lines closer and closer to being parallel, the point
at which they cross gets farther and farther away, and eventually becomes infinitely distant.}
\scanshdr{realisupdown}{First let's use the front-back method. The
reflected rays are going to the left, so an observer would have to be standing on the left in order
to see the image. Rays from the face can get to the mirror, but the rays from the back of
the head are blocked by the head. If the image-face is visible from the left, then the image-head
must be facing to the left, which is the way it's shown in figure \figref{inbendingmirrorreal}.
The nose on the real face is the face's closest point to the mirror, but on the image-face it's the
farthest from the mirror. Therefore the image is $\updown$: greater object distances result in
smaller image distances. You can also easily verify this result using a ray diagram. As the
object moves farther from the mirror, the incident and reflected rays fan out from the normal.}
\scanshdr{realmagnification}{From the ray diagram, we can see that the distance from the image
to the mirror is less than the distance from the object to the mirror. The other distances are
in the same proportion.}



